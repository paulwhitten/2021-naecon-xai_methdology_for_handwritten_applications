\documentclass[conference]{IEEEtran}
\IEEEoverridecommandlockouts
% The preceding line is only needed to identify funding in the first footnote. If that is unneeded, please comment it out.
\usepackage{cite}
\usepackage{amsmath,amssymb,amsfonts}
\usepackage{algorithmic}
\usepackage{graphicx}
\usepackage{textcomp}
\usepackage{xcolor}
\usepackage{array}
\def\BibTeX{{\rm B\kern-.05em{\sc i\kern-.025em b}\kern-.08em
    T\kern-.1667em\lower.7ex\hbox{E}\kern-.125emX}}
\begin{document}

\title{Explainable Artificial Intelligence \\ {Methodology for Handwritten Applications}}

\author{\IEEEauthorblockN{Paul Whitten, Francis Wolff, Chris Papachristou}
\IEEEauthorblockA{\textit{Electrical, Computer, and Systems Engineering} \\
\textit{Case School of Engineering} \\
\textit{Case Western Reserve University} \\
Cleveland, OH, USA \\
pcw@case.edu, fxw12@case.edu, cap2@case.edu}

}

\maketitle

\begin{abstract}
There has been explosive growth of practical AI in recent years.
%However, inferential results of AI systems are not readily explainable to humans.
A major concern of current AI systems is an inability to explain inferential decisions.
%Explainable artificial intelligence has been posed to mitigate these concerns.
This work explores an Explainable Artificial Intelligence (XAI) methodology that provides explanations
for classification decisions.  Experimental results using the MNIST handwritten digit database are provided with explainable conclusions.
\end{abstract}

\begin{IEEEkeywords}
explainable, artificial intelligence, machine learning
\end{IEEEkeywords}

\section{Introduction}

Recent advances in Machine Learning (ML) have brought about wide adoption of ML algorithms for many applications.  Despite various successes, there is a reluctance to adopt ML in some applications because ML behaves like a black box.  Decision making by the black box is often not explainable to humans. 

This work approaches the widely studied problem of classifying images of handwritten digits into the ten decimal digit classes, zero through nine, from an Explainable Artificial Intelligence (XAI) perspective.  Our goal is to provide explainable ML classification, in the form of rationale, for classification decisions.  This is achieved using an architecture, with fine-grained classification decisions based on properties, and a methodology for constructing an explainable architecture.  Explainability is designed into and is an active component of our proposed architecture\cite{Arrieta2020ExplainableAI}. 

Using the MNIST handwritten digit database, we apply the methodology and architecture.  The approach is outlined herein along with test cases and examples of explainability.  Finally we present results and metrics to gauge accuracy and explainability.

This work aims to pose a means of explaining classification to a human.  We do not wish to compete with established algorithms that perform exceptionally well in classification of input \cite{keysers07} \cite{lecun98} \cite{schm2012}.  The effort required in applying the methodology in this work is significant compared to training a classifier that will act as a black box, and therefore not be explainable.

While we approach XAI for a specific classification problem, in the MNIST handwritten digit database, the methodology translates to other challenges requiring explainable classification among a finite set of classes.

\section{Related work}

The ability to map the learning classifier or recognizer to human-based explainability is a challenging task for human understandability.  Currently, there are at least seventeen explainable techniques such as
decision tree-based, rule-based (i.e. knowledgebase), salience mapping,
sensitivity-based analysis, feature importance, fuzzy-based, neural-network, and generic-programming based.  These techniques use one of three basic evaluation approaches: application-grounded, human-grounded and functionally grounded. \cite{BlackBox18} \cite{Arrieta2020ExplainableAI} \cite{Survey18} \cite{Fuzzy19} \cite{Hagras18}  \cite{GP18}

Distributed and fault tolerant systems research has provided several examples of voting \cite{avizienis} and probabilistic models for the voting problem \cite{blough}.

\section{Overview}

 \begin{figure}[htbp]
\centerline{\includegraphics[width=95mm]{./images/voting_prop_nn_2.png}}
\caption{XAI Architecture Summary}
\label{voting}
\end{figure}

The methodology for explainability is based upon the use of explainable properties and related transformations of the input to make distinct classification decisions for each property.  Those distinct classifications are input to a voter to provide a classification decision.  Rationale for the decision is composed from explainable property classifications and combined with the voter result to provide an explanation to a user. 

The explainable architecture used in this work is depicted in Fig.~\ref{voting}.  $I_i$ represents the $i$th input to the architecture.  The architecture consists of Properties, a Voter Model Engine (VME),  and an Explainable Artificial Intelligence block.   Explainable properties are defined as descriptive qualities of a sample input that mean something to a user in the problem domain, especially to justify a classification decision.   Explainable properties $P_1$ through $P_n$ are outlined in the blue Property rectangle.  Each property represents logic for a classification based solely on the property.   $Q_i$ represents the $i$th property's classification output.

The VME relies on classifications from the explainable properties, $Q_i$, and feedback from the XAI block to make a final classification decision.   We explore two voting model schemes detailed in section \ref{subsection:Voting}.

The XAI block consists of a knowledgebase and logic to generate an explanation for the user.  Inputs to the XAI block are explainable property classifications, $Q_i$, and the VME decision.  The XAI knowledgebase contains information on explainable properties,  a store of training results, and metrics related to the effectiveness of properties.  The explanation provided by the XAI block consists of rationale that relate to the explainable properties contributing to the classification decision.

 \begin{figure}[htbp]
\centerline{\includegraphics[width=90mm]{./images/property_transforms.png}}
\caption{Architecture of a single property for multiple transformations.}
\label{proptrans}
\end{figure} 

A property transformation is a modifying function applied to the input.  Transformations are used to identify the explainable property in the input.  Each property may have one or more transformations.   This is represented in Fig.~\ref{proptrans} where the outer square represents the $i$th property from Fig.~\ref{voting}.  The boxes labeled $T_x$ indicate the $x$th transformation of the input.  Transformed input is fed to a trained Neural Network Architecture (NNA) to make classification decisions.  Classification output then flows to the VME as shown in Fig.~\ref{voting}.

\section{Methodology}
 
Our methodology for achieving explainable ML classification involves the following steps:
%\begin{itemize}
%\item Discover explainable properties.
%\item Define transformations for explainable properties.
%\item Transform training data.
%\item Produce trained explainable property-specific NNAs.
%\item Build a Knowledgebase across the explainable properties.
%\item Devise a voting scheme.
%\item Use a test dataset to provide feedback.
%\end{itemize}

\subsubsection{Discover Explainable Properties}
An explainable property is an attribute of a sample in the problem domain that may differentiate classes and provide a rationale for classification to a a user.   An explainable property need not be present in all classes.  Discovering explainable properties may require manual analysis of sample inputs.

\subsubsection{Define and Implement Transformations}
Data transformations are next defined and implemented to highlight explainable properties in the input.  Transforms may be known algorithms of feature detection and extraction that relate to explainable properties.

\subsubsection{Transform Training Data} 
We next generate a transformed training dataset by submitting all elements from the training set to the property transformations.  The output from property transformations are stored for training the property transform specific NNAs. 

\subsubsection{Produce Trained Property NNAs}
The next step involves initializing unique NNAs for each property transformation.  Using supervised ML techniques, the NNAs are trained from the data stored from the previous step.  This results in trained NNAs that produce classifications for explainable property transforms.

\subsubsection{Build an XAI Knowledgebase}
After training, we again process the training set and populate the XAI knowledgebase with the property classification results.  The knowledgebase also stores information on explainable properties, used for composing rationale, and each property's classification result.  Metrics on the per-class effectiveness of each explainable property are also calculated and stored in the knowledgebase for use by the VME.

\subsubsection{Devise a Voting Scheme}
We next devise a voting scheme in the VME.  The purpose for the voting scheme is to select among the potentially conflicting votes from the explainable property transformation classifications.  Voting decisions are based on information from the knowledgebase and explainable property classifications.  We discuss probabilistic and ML based voting schemes later in section \ref{subsection:Voting}.

\subsubsection{Provide User Feedback}
Finally, when test data is presented to the architecture, we evaluate the results and determine if they are sufficient.  The performance of the explainable property transforms for particular classes are examined for effectiveness.  Where there are gaps, new properties are identified for classes with poor results.  Ineffective properties or transforms may also be eliminated.

\section{Approach}

This section presents the approach of applying the methodology to the MNIST handwritten digit database. 

\subsection{Explainable Properties}

Through manual review of samples in MNIST, we identified explainable properties related to shapes and characteristics of the digits.  The explainable properties we identified and utilized are listed in the Property column of Fig.~\ref{transsample}.

\subsection{Transformations}
 
Digital image processing techniques that relate to and highlight the explainable properties were used as transforms.  The Transform column in Fig.~\ref{transsample} lists transformations for the various properties.  The figure also shows example MNIST digits, $I_i$, and their resulting transformed images, $T_i$.

\bgroup
\renewcommand{\arraystretch}{1.8}
%\setlength\tabcolsep{2mm}
\begin{figure}
\centering
\resizebox{\columnwidth}{!}{%
\begin{tabular}{ c | p{0.23\linewidth} | p{0.23\linewidth} | ccc | }
\cline{2-6}
& Property & Transform & $I_i$ &  &  $T_i$ \\
\hline \hline
$P_1$ & Stroke & Skeleton & \raisebox{-.5\height}{\includegraphics[width=8mm]{./digit-images/4-11.png}} & $\rightarrow$ & \raisebox{-.5\height}{\includegraphics[width=8mm]{./digit-images/4-11-skel.png}} \\
\hline
$P_2$ & Circle & Hough Circle & \raisebox{-.5\height}{\includegraphics[width=8mm]{./digit-images/6-17.png}} & $\rightarrow$ & \raisebox{-.5\height}{\includegraphics[width=8mm]{./digit-images/6-17-circle.png}} \\
\hline
$P_3$ & Crossings & Crossings & \raisebox{-.5\height}{\includegraphics[width=8mm]{./digit-images/4-2.png}} & $\rightarrow$ & \raisebox{-.5\height}{\includegraphics[width=8mm]{./digit-images/4-2-crossing.png}} \\
\hline
$P_4$ & Circle & Hough Ellipse & \raisebox{-.5\height}{\includegraphics[width=8mm]{./digit-images/0-3.png}} & $\rightarrow$ & \raisebox{-.5\height}{\includegraphics[width=8mm]{./digit-images/0-3-ellipse.png}} \\
\hline
$P_5$ & Circle & Multiple Ellipse Circle & \raisebox{-.5\height}{\includegraphics[width=8mm]{./digit-images/8-4.png}} & $\rightarrow$ & \raisebox{-.5\height}{\includegraphics[width=8mm]{./digit-images/8-4-ellipse-circle.png}} \\
\hline
$P_6$ & Endpoints & Endpoints & \raisebox{-.5\height}{\includegraphics[width=8mm]{./digit-images/2-2.png}} & $\rightarrow$ & \raisebox{-.5\height}{\includegraphics[width=8mm]{./digit-images/2-2-endpoint.png}} \\
\hline
$P_7$ & Enclosed Region & Flood Fill & \raisebox{-.5\height}{\includegraphics[width=8mm]{./digit-images/0-2.png}} & $\rightarrow$ & \raisebox{-.5\height}{\includegraphics[width=8mm]{./digit-images/0-2-fill.png}} \\
\hline
$P_8$ & Line & Hough Line & \raisebox{-.5\height}{\includegraphics[width=8mm]{./digit-images/7-20.png}} & $\rightarrow$ & \raisebox{-.5\height}{\includegraphics[width=8mm]{./digit-images/7-20-line.png}} \\
\hline
$P_9$ & Enclosed Region & Skeleton Flood Fill & \raisebox{-.5\height}{\includegraphics[width=8mm]{./digit-images/8-3.png}} & $\rightarrow$ & \raisebox{-.5\height}{\includegraphics[width=8mm]{./digit-images/8-3-skel-fill.png}} \\
\hline
\end{tabular}%
}
\centering
\caption{Properties and transforms for the MNIST example}
\label{transsample}
\end{figure}
\egroup

The Stroke property is meant to represent the minimal path of the writing implement to trace the digit.  The morphological skeleton transformation is a one pixel connected representation of the digit,  representing the stroke.  We utilized the Lee\cite{Lee1994} algorithm for the skeleton.
$P_2$, $P_4$, and $P_5$ are the circle property with corresponding transforms $T_2$, $T_4$, and $T_5$ representing the Hough Circle, the Hough Ellipse, and multiple non-overlapping circles and ellipses.
$P_3$ and $T_3$ are the crossings property and transform representing the intersection of line segments in a digit.  $T_3$ involves taking the skeleton and then finding activated pixels with more than two neighbors.   The endpoints property and transform, $P_6$ and $T_6$, involved taking activated pixels in the skeleton with only one neighbor.
Property $P_7$ and $P_9$ for enclosed regions used transform $T_7$ which involved a flood fill and $T_9$ the flood fill of the skeleton.  The line property and transform, $P_8$ and $T_8$, uses an algorithm to find the non-overlapping Hough Lines of a digit.

\subsection{Transforming Training Data}

The various transforms were applied to the MNIST data using implementations in the Python scikit-image\ref{scikit-image} library version 0.17.2.  The resulting transformed images were stored in MNIST IDX files for each property transform.

\subsection{Training}

The trained NNAs were implemented using the Python scikit-learn\ref{scikit-learn} version 0.23.2 Multi-Layer Perceptrons.  The transformed data stored in the MNIST IDX files for each property transform were used to train the NNAs.

\subsection{Knowledgebase}

After training the NNAs,  the transformed training data was presented to the NNAs and results were stored in a data structure that was serialized as JSON.  The JSON could be loaded to in-memory maps for performing metric calculation and efficient runtime access.  Textual property labels and descriptions were also stored in the knowledgebase for composing explainable rationale for classification.

\subsection{Voting Schemes}
\label{subsection:Voting}

The first voting scheme we present is probabilistic.  In this scheme, we use the knowledgebase to identify each explainable property transformation's effectiveness to correctly predict a digit.   The effectiveness for an explainable property transformation, $i$, to select a particular digit, $x$,  is given by:
\begin{equation}\label{effectiveness}
E_{i,x}  = \frac{|A|}{|B|}
\end{equation}
where $A$ is the set of correct classifications of explainable property transformation $i$ of the digit $x$ and $B$ is the set of elements of digit $x$.  The weight of the effectiveness for each digit, $x$, is:
\begin{equation}\label{weight}
W_x=\sum_i E_{i, x}
\end{equation}
The confidence for the digit $x$ is calculated by the VME and is given by:
\begin{equation}\label{conf}
C_x=\frac{W_x}{\sum\limits_jW_j}
\end{equation}
The denominator is the sum of all weights for classes that were selected by a property.  If multiple digits are selected by the properties, the digit with the highest confidence will be chosen by the VME.

The second voting scheme was NN based.  We utilized explainable property transformation output and labels from MNIST, stored in the database, to train a Multi-Layer Perceptron model in the NN VME.

\subsection{ User Feedback}

When evaluating early results, there was particular difficulty with the circle transforms for some digits.   The single circle transforms worked well for detecting digits six and nine but performed rather poorly for eight and zero.  The ellipse proved better in extracting characteristics handwritten zeros.   Adding a transform that used multiple non-overlapping circles or ellipses was also an improvement for the digit eight.  We experimented with the inflection point property and corner detection transforms but the classification results were poor for all digit classes so the property was eliminated.

\section{Examples}

Detailed examples from MNIST follow in this section by first providing aggregate property classification results for the digits five and six and then explainable results from three different input images.

Tables ~\ref{table:digit5out} and ~\ref{table:digit6out} show the property transformation NNA output and statistics for properties for digits five and six.  The tables represent the results from reprocessing all of the training digits labeled five and six.  Columns two through ten of the tables represent the property transformations, $P_i$ from Fig.~\ref{transsample}.  The first row of the tables represents effectiveness, $E_{i,x}$, where $i$ is the property transform and $x$ is the digit.  The second, third and fourth rows represent the standard deviation, kurtosis, and skew of the digit outputs for each property.  The numbered rows in the tables represent the means of the property outputs for each digit.  The last two rows are the false positive and false negative rates.

We observe that the Stroke property, $P_0$, performed very well in both digits.  The next highest performing property in both tables was the endpoint property, $P_5$, with 85.4\% and 93.3\% accuracy.  The digit six also had good classification results for the enclosed region properties, $P_7$ and $P_9$.  The digit six also performed better than the five in the properties related to the circle, $P_2$, $P_4$, and $P_5$.  The digit five had among the poorest performance as observed from the relatively low percent correct and high false negative rates.

\begin{table}
\caption{Digit 5 Outputs}
\centering
\resizebox{\columnwidth}{!}{%
\begin{tabular}{ | c ||  c | c | c | c | c | c | c | c | c |}
Digit 5 & $P_1$ & $P_2$ & $P_3$ & $P_4$ & $P_5$ & $P_6$ & $P_7$ & $P_8$ & $P_9$ \\
\hline \hline
$E_{i,5}$  & 100.0 & 14.2 & 5.7 & 20.8 & 21.5 & 85.4 & 3.8 & 70.0 & 5.5 \\
\hline
$\sigma$ & 0.300& 0.076& 0.082& 0.068& 0.089& 0.252& 0.074& 0.211& 0.077 \\
\hline
k & 10.000& 3.995& -1.864& 5.555& 6.229& 9.550& -1.941& 9.779& -2.063 \\
\hline
skew & 3.162& 1.913& 0.330& 2.324& 2.413& 3.073& -0.025& 3.116& 0.061 \\
\hline
0 & 0.000 & 0.115 & 0.208 & 0.093 & 0.073 & 0.002 & 0.007 & 0.023 & 0.020 \\
\hline
1 & 0.000 & 0.048 & 0.223 & 0.057 & 0.044 & 0.111 & 0.193 & 0.008 & 0.183 \\
\hline
2 & 0.000 & 0.048 & 0.051 & 0.058 & 0.053 & 0.001 & 0.089 & 0.027 & 0.069 \\
\hline
3 & 0.000 & 0.162 & 0.082 & 0.154 & 0.154 & 0.002 & 0.149 & 0.079 & 0.178 \\
\hline
4 & 0.000 & 0.046 & 0.006 & 0.056 & 0.045 & 0.003 & 0.122 & 0.019 & 0.130 \\
\hline
5 & 1.000 & 0.300 & 0.200 & 0.284 & 0.345 & 0.849 & 0.186 & 0.732 & 0.185 \\
\hline
6 & 0.000 & 0.097 & 0.069 & 0.059 & 0.090 & 0.009 & 0.027 & 0.036 & 0.037 \\
\hline
7 & 0.000 & 0.045 & 0.167 & 0.067 & 0.041 & 0.001 & 0.185 & 0.011 & 0.212 \\
\hline
8 & 0.000 & 0.109 & 0.022 & 0.094 & 0.101 & 0.014 & 0.005 & 0.042 & 0.005 \\
\hline
9 & 0.000 & 0.044 & 0.019 & 0.069 & 0.043 & 0.009 & 0.032 & 0.034 & 0.028 \\
\hline
false pos. \%  & 0.0 & 0.5 & 0.2 & 0.4 & 0.7 & 0.4 & 0.0 & 0.3 & 0.0 \\
\hline
false neg. \%  & 0.0 & 84.8 & 94.1 & 78.8 & 78.3 & 13.4 & 96.2 & 29.2 & 94.5 \\
\hline
\end{tabular}%
}
\label{table:digit5out}
\end{table}

\begin{table}
\caption{Digit 6 Outputs}
\centering
\resizebox{\columnwidth}{!}{%
\begin{tabular}{ | c ||  c | c | c | c | c | c | c | c | c |}
Digit 6 & $P_1$ & $P_2$ & $P_3$ & $P_4$ & $P_5$ & $P_6$ & $P_7$ & $P_8$ & $P_9$ \\
\hline \hline
$E_{i,6}$  & 100.0 & 46.5 & 49.5 & 32.0 & 49.0 & 93.3 & 81.8 & 70.7 & 83.2 \\
\hline
$\sigma$ & 0.300& 0.112& 0.130& 0.094& 0.133& 0.264& 0.240& 0.209& 0.246 \\
\hline
k & 10.000& 8.507& 8.617& 9.864& 9.469& 9.955& 9.935& 9.889& 9.930 \\
\hline
skew & 3.162& 2.842& 2.886& 3.132& 3.048& 3.153& 3.148& 3.138& 3.147 \\
\hline
0 & 0.000 & 0.100 & 0.042 & 0.069 & 0.080 & 0.020 & 0.001 & 0.034 & 0.004 \\
\hline
1 & 0.000 & 0.052 & 0.050 & 0.066 & 0.048 & 0.005 & 0.035 & 0.011 & 0.033 \\
\hline
2 & 0.000 & 0.057 & 0.135 & 0.064 & 0.061 & 0.020 & 0.022 & 0.028 & 0.013 \\
\hline
3 & 0.000 & 0.095 & 0.046 & 0.070 & 0.079 & 0.002 & 0.027 & 0.051 & 0.032 \\
\hline
4 & 0.000 & 0.039 & 0.044 & 0.067 & 0.040 & 0.000 & 0.027 & 0.046 & 0.023 \\
\hline
5 & 0.000 & 0.101 & 0.065 & 0.052 & 0.086 & 0.008 & 0.028 & 0.029 & 0.024 \\
\hline
6 & 1.000 & 0.426 & 0.482 & 0.380 & 0.495 & 0.890 & 0.818 & 0.725 & 0.837 \\
\hline
7 & 0.000 & 0.057 & 0.049 & 0.077 & 0.047 & 0.000 & 0.033 & 0.007 & 0.038 \\
\hline
8 & 0.000 & 0.027 & 0.086 & 0.068 & 0.030 & 0.034 & 0.004 & 0.041 & 0.001 \\
\hline
9 & 0.000 & 0.029 & 0.026 & 0.077 & 0.038 & 0.000 & 0.007 & 0.027 & 0.005 \\
\hline
false pos. \%  & 0.0 & 2.6 & 2.5 & 0.5 & 2.2 & 0.7 & 0.2 & 0.6 & 0.0 \\
\hline
false neg. \%  & 0.0 & 53.3 & 50.1 & 67.5 & 50.7 & 6.6 & 18.1 & 28.1 & 16.8 \\
\hline
\end{tabular}%
}
\label{table:digit6out}
\end{table}

Mean digit values from the Property NNAs are also represented  in Fig.~\ref{digit5votes} and ~\ref{digit6votes} as three dimensional surface plots for digits five and  and six.

\begin{figure}[htbp]
\begin{minipage}{0.48\textwidth}
\centerline{\includegraphics[width=50mm]{./images/digit-5.png}}
\caption{Mean property NNA output for the digit 5}
\label{digit5votes}
\end{minipage}\hfill
\begin{minipage}{0.48\textwidth}
\centerline{\includegraphics[width=50mm]{./images/digit-6.png}}
\caption{Mean property NNA output for the digit 6}
\label{digit6votes}
\end{minipage}
\end{figure}

%\begin{table}[htbp]
%\caption{Properties, Transforms and Property Identifiers}
%\centering
%\begin{tabular}{| c | c | c |}
%\hline
% Identifier & Explainable Property & Transform \\
%\hline\hline
%$P_1$ & Stroke & Skeleton \\
%\hline
%$P_x$ & Circle & Hough Circle \\
%\hline
%$P_x$ & Crossings & Crossing Point \\
%\hline
%$P_x$ & Ellipse & Hough Ellipse \\
%\hline
%$P_x$ & Ellipse + Circle & Hough Ellipse and Circle \\
%\hline
%$P_x$ & Endpoints & Endpoints \\
%\hline
%$P_x$ & Enclosed Region & Flood Fill \\
%\hline
%$P_x$ & Line & Hough Line \\
%\hline
%$P_x$ & Enclosed Region of Skeleton & Skeleton Flood Fill \\
%\hline
%\end{tabular}
%\label{table:tblproptrans}
%\end{table}

 \begin{figure}[htbp]
\centerline{\includegraphics[width=15mm]{./digit-images/5-0.png}}
\caption{Example 1 of a handwritten digit five}
\label{example1}
\end{figure}

We next present three interesting MNIST digits and review the property classifications, property effectiveness,  confidence, and explainability.  Stepping through the procedures of the probabilistic voting scheme, we show the classification results for the digits as well as the explainability rationale from XAI.

\begin{table}[htbp]
\caption{Probabilistic voting and explainability for Example 1}
\centering
\begin{tabular}{| c | c | c | c | p{0.08\linewidth} | p{0.08\linewidth} |}
\cline{3-6}
\multicolumn{2}{c}{} & \multicolumn{2}{|c|}{Effectiveness} & \multicolumn{2}{c|}{Explainability} \\
\hline
 Prop. & Vote & $E_{i,5}$ & $E_{i,6}$ & $X_5$ & $X_6$ \\
\hline \cline{0-5}
$P_1$ & 5 & 1.000 & - & \checkmark & - \\ 
\hline
$P_2$ & 6 & - & 0.465 & - & \checkmark \\
\hline
$P_3$ & - & - &  - & - & - \\
\hline
$P_4$ & - & - & - & - & - \\
\hline
$P_5$ & 6 & - & 0.490 & - & \checkmark \\
\hline
$P_6$ & 5 & 0.854 & - & \checkmark & - \\
\hline
$P_7$ & - & - & - & - & - \\
\hline
$P_8$ & 5 & 0.700 & - & \checkmark & - \\
\hline
$P_9$ & - & - & - & - & - \\
\hline \cline{0-5}
\multicolumn{2}{|c|}{Weight Totals} & $2.554$ & $0.955$ & \multicolumn{2}{c|}{$\sum W_\gamma=3.509$} \\
\cline{0-5}
\multicolumn{2}{|c|}{Confidence} & $73\%$ & $24\%$ & \multicolumn{2}{c}{} \\
\cline{0-3}
\end{tabular}
\label{table:example1}
\end{table}

The first example digit, labeled a five, is shown in Fig.~\ref{example1}.  Voting and explainability is detailed in Table ~\ref{table:example1}.  The property votes for this example are shown in the Vote column.  Cells with dashes are properties that did not have a sufficiently strong opinion on a particular digit.  I.e., no digit was above a threshold for that property.   The $E_{i,x}$ columns give the effectiveness of a property $i$ to correctly select class $x$.  The effectiveness values are from the knowledgebase using eq. (\ref{effectiveness}).   The stroke $(P_1)$, endpoint $(P_6)$ and line $(P_8)$ properties suggest the digit is a five with effectiveness $E_{1,5}= 1.000$, $E_{6,5}=0.854$, and $E_{8,5}=0.700$.  Weight of effectiveness for the digit five is $W_5=2.554$, given by eq. (\ref{weight}).  The circle $(P_2)$ and $(P_5)$ properties suggest that the digit is a six with effectiveness $E_{2,6}=0.465$ and $E_{5,6}=0.490$  and a weight of $W_6=0.955$.  Taking the sum of all weights, $\sum\limits_i W_i=3.509$.  Confidence, as given by eq. (\ref{conf}), for the five, is given as $C_5=\frac{2.554}{3.509} = 73\%$.  Alternatively, six was suggested by properties where confidence is $C_6=\frac{0.955}{3.509}=27\%$.  Five wins because $C_6=27\% < 73\%=C_5$.  We also observe that the explainability, $X_x$, columns provide rationale for each of the digits that was selected.  The logic in the XAI block references property information from explainability column $X_5$, corresponding to the winning digit, and assembles rationale indicating that the five was selected because the stroke, endpoint, and line properties are consistent with a five. 

 \begin{figure}[htbp]
\centerline{\includegraphics[width=15mm]{./digit-images/9-9.png}}
\caption{Example 2 of a handwritten digit nine}
\label{example2}
\end{figure}

%\begin{table}[htbp]
%\caption{Property Votes for Example 2}
%\centering
%\begin{tabular}{| c | c | c |}
%\hline
% Property Id & Vote & Weight \\
%\hline\hline
%$P_0$ & 9 & 1.000 \\ 
%\hline
%$P_1$ & 3 & 0.327 \\
%\hline
%$P_2$ & 5 & 0.057 \\
%\hline
%$P_3$ & - & - \\
%\hline
%$P_4$ & - & - \\
%\hline
%$P_5$ & 6 & 0.932 \\
%\hline
%$P_6$ & 9 & 0.809 \\
%\hline
%$P_7$ & - & - \\
%\hline
%$P_8$ & 9 & 0.821 \\
%\hline
%\end{tabular}
%\label{table:example2}
%\end{table}

The second example,  labeled a nine, is shown in Fig.~\ref{example2}.  Table ~\ref{table:example2} shows the property predictions, voting results, and rationale for example 3.  In this example three properties selected the digit nine and three other properties selected the digits three, five, and six.  The results from the VME for this example were that nine was selected with a $67\%$ confidence and an explanation that the stroke and enclosed region properties were consistent with those of a nine.  We noted on this example that it is surprising that none of the properties indicated an eight because of the example's similarity to a digit eight.  

\begin{table}[htbp]
\caption{Probabilistic voting and explainability for Example 2}
\centering
\resizebox{\columnwidth}{!}{%
\begin{tabular}{| c | c | c | c | c | c | c | c | c | c |}
\cline{3-10}
\multicolumn{2}{c}{} & \multicolumn{4}{|c|}{Effectiveness} & \multicolumn{4}{c|}{Explainability} \\
\hline
 Prop. & Vote & $E_{i,3}$ & $E_{i,5}$ & $E_{i,6}$ & $E_{i,9}$ & $X_3$ & $X_5$ & $X_6$ & $X_9$ \\
\hline \cline{0-9}
$P_1$ & 9 & - & - & - & 1.00 & - & - & - & \checkmark \\ 
\hline
$P_2$ & 3 & 0.327 & - & - & - & \checkmark & - & - & - \\
\hline
$P_3$ & 5 & - &  0.057 & - & - & - & \checkmark & - & - \\
\hline
$P_4$ & - & - & - & - & - & - & - & - & - \\
\hline
$P_5$ & - & - & - & - & - & - & - & - & - \\
\hline
$P_6$ & 6 & - & - & 0.933 & - & - & - & \checkmark & - \\
\hline
$P_7$ & 9 & - & - & - & 0.809 & - & - & - & \checkmark \\
\hline
$P_8$ & - & - & - & - & - & - & - & - & - \\
\hline
$P_9$ & 9 & - & - & - & 0.821 & - & - & - & \checkmark \\
\hline \cline{0-9}
\multicolumn{2}{|c|}{Weight} & 0.327 & 0.057 & 0.933 & 2.630 & \multicolumn{4}{c|}{$\sum W_\gamma=3.946$} \\
\cline{0-9}
\multicolumn{2}{|c|}{Confidence} & $8\%$ & $1\%$ & $24\%$ & $67\%$ & \multicolumn{4}{c}{} \\
\cline{0-5}
\end{tabular}%
}
\label{table:example2}
\end{table}

 \begin{figure}[htbp]
\centerline{\includegraphics[width=15mm]{./digit-images/2-4.png}}
\caption{Example 3 of a handwritten digit two}
\label{example3}
\end{figure}

%\begin{table}[htbp]
%\caption{Property Votes for Example 3}
%\centering
%\begin{tabular}{| c | c | c |}
%\hline
% Property Id & Vote & Weight \\
%\hline\hline
%$P_0$ & 2 & 1.000 \\ 
%\hline
%$P_1$ & 3 & 0.327 \\
%\hline
%$P_2$ & - & - \\
%\hline
%$P_3$ & 2 & 0.161 \\
%\hline
%$P_4$ & 3 & 0.387 \\
%\hline
%$P_5$ & 2 & 0.938 \\
%\hline
%$P_6$ & - & - \\
%\hline
%$P_7$ & 2 & 0.639 \\
%\hline
%$P_8$ & - & - \\
%\hline
%\end{tabular}
%\label{table:example3}
%\end{table}

\begin{table}[htbp]
\caption{Probabilistic voting and explainability for Example 3}
\centering
\begin{tabular}{| c | c | c | c | p{0.08\linewidth} | p{0.08\linewidth} |}
\cline{3-6}
\multicolumn{2}{c}{} & \multicolumn{2}{|c|}{Effectiveness} & \multicolumn{2}{c|}{Explainability} \\
\hline
 Prop. & Vote & $E_{i,2}$ & $E_{i,3}$ & $X_2$ & $X_3$ \\
\hline \cline{0-5}
$P_1$ & 2 & 1.000 & - & \checkmark & - \\ 
\hline
$P_2$ & 3 & - & 0.327 & - & \checkmark \\
\hline
$P_3$ & - & - &  - & - & - \\
\hline
$P_4$ & 2 & 0.161 & - & \checkmark & - \\
\hline
$P_5$ & 3 & - & 0.387 & - & \checkmark \\
\hline
$P_6$ & 2 & 0.938 & - & \checkmark & - \\
\hline
$P_7$ & - & - & - & - & - \\
\hline
$P_8$ & 2 & 0.639 & - & \checkmark & - \\
\hline
$P_9$ & - & - & - & - & - \\
\hline \cline{0-5}
\multicolumn{2}{|c|}{Weight Totals} & $2.738$ & $0.714$ & \multicolumn{2}{c|}{$\sum W_\gamma=3.452$} \\
\cline{0-5}
\multicolumn{2}{|c|}{Confidence} & $79\%$ & $21\%$ & \multicolumn{2}{c}{} \\
\cline{0-3}
\end{tabular}
\label{table:example3}
\end{table}

The third example handwritten digit, labeled a two, is shown in Fig. ~\ref{example3} and Table ~\ref{table:example3}.  In this case,  the VME selects the digit two with a $79\%$ confidence due to stroke, circle, endpoint, and line properties consistent with the digit two.  A three was suggested with $21\%$ confidence due to circle properties.  We noted in this example that classification results from circle properties had conflicting votes.  $P_2$ and $P5$ voted for three while $P_4$ voted for two.

\begin{table}[htbp]
\caption{NNA VME results for examples}
\centering
\begin{tabular}{| c | c | c | c |}
\hline
 Digit & Ex. 1 & Ex. 2 & Ex. 3 \\
\hline\hline
0 & 1.92e-06 & 1.11e-08 & 2.98e-07\\ 
\hline
1 & 1.44e-06 & 1.52e-07 & 1.93e-08 \\
\hline
2 & 1.40e-06 & 4.54e-09 & \textbf{9.99e-01} \\
\hline
3 & 5.12e-08 & 2.48e-04 & 5.26e-06 \\
\hline
4 & 1.79e-08 & 1.33e-06 & 5.91e-06 \\
\hline
5 & \textbf{9.99e-01} & 2.31e-07 & 1.41e-06 \\
\hline
6 & 7.69e-07 & 2.23e-04 & 5.83e-07 \\
\hline
7 & 4.88e-07 & 4.93e-08 & 1.89e-09 \\
\hline
8 & 2.47e-07 & 5.71e-06 & 5.51e-07 \\
\hline
9 & 5.21e-07 & \textbf{9.99e-01} & 1.71e-08 \\
\hline
\end{tabular}
\label{table:nnavoter}
\end{table}

Table~\ref{table:nnavoter} shows the output of the ML voting scheme on the three examples.  The ten rows in the table represent the corresponding digits.  The columns for examples 1 through 3 contain the values output by the NN VME when presented with the property votes from the examples.  We observe that in each example the NN VME overwhelmingly selects the appropriate digit shown in bold.

\section{Results}

We introduce two metrics to compare aggregate results from the two voting schemes.  The first metric is Identification Accuracy which gauges the correctness of the system in performing classification.  Identification Accuracy is the ratio of the number of correct classifications to the total number of classifications of the VME  The second metric, Explainability Quality, is used to estimate the correspondence or connection of explanations with classification decisions.  Explainability Quality is the ratio of classifications decisions with at least one property justifying the classification by the voting scheme to the total number of classifications.   In both metrics, values close to one are desired.

Results for Identification Accuracy obtained using the probabilistic voting scheme on the MNIST dataset were $0.919$ while results obtained from the NNA voting scheme were $0.959$.  Results for Explainability Quality using the probabilistic voting scheme were $1.00$ while results for the NNA voting scheme were $0.982$.

The ML voting scheme was more accurate in classifying input than the probabilistic scheme by about 4\%.   However, the probabilistic VME was nearly 2\% better in terms of Explainability Quality over the ML VME.  Note that in the probabilistic scheme, the VME selects a class from among the property votes, so there is always a property that corresponds to the VME output.  The NN in the ML voting scheme is trained based on labels on the training set which may not correspond to a vote from a property in a small percentage of inputs.

It appears there can be cases where identification accuracy and explanation quality may not be well balanced.  The particulars of the application will need to be considered.  It is our view that, in some applications, the quality of explainability may have greater importance than identification accuracy.

%\section{References}
\bibliographystyle{plain}
\bibliography{references}{}


\end{document}
