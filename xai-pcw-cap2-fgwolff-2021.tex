\documentclass[conference]{IEEEtran}
\IEEEoverridecommandlockouts
% The preceding line is only needed to identify funding in the first footnote. If that is unneeded, please comment it out.
\usepackage{cite}
\usepackage{amsmath,amssymb,amsfonts}
\usepackage{algorithmic}
\usepackage{graphicx}
\usepackage{textcomp}
\usepackage{xcolor}
\def\BibTeX{{\rm B\kern-.05em{\sc i\kern-.025em b}\kern-.08em
    T\kern-.1667em\lower.7ex\hbox{E}\kern-.125emX}}
\begin{document}

\title{Property-Based Explainable Artificial Intelligence for Handwritten Digit Classification}

\author{\IEEEauthorblockN{Paul Whitten, Francis Wolff, Chris Papachristou}
\IEEEauthorblockA{\textit{Electrical Engineering and Computer Science} \\
\textit{Case Western Reserve University} \\
Cleveland, OH, USA \\
0000-0002-7787-473X, cap2@case.edu, fxw12@case.edu}
}

\maketitle

\begin{abstract}
There has been explosive growth of practical AI in recent years.
However, results of AI systems are not readily explainable to humans.
A major concern of current AI systems is an inability to explain decisions.
Explainable artificial intelligence has been posed to mitigate these concerns.
This work is an attempt to explore a methodology that provides explanations
for classification decisions.
\end{abstract}

\begin{IEEEkeywords}
explainable, artificial intelligence, machine learning
\end{IEEEkeywords}

\section{Introduction}

Recent advances in Machine Learning (ML) have brought about wide adoption of ML algorithms for many applications.  Despite various successes, there is a reluctance to adopt ML in some applications because ML behaves like a black box, decision making by the black box is often not explainable to humans.  This working paper approaches the widely studied problem of classifying images of handwritten digits into the ten decimal digit classes, zero through nine, from an Explainable Artificial Intelligence (XAI) perspective.

While we approach XAI for a specific classification problem in the MNIST handwritten digit database, which consists of 28x28 images of handwritten digits into the ten decimal digits, we feel the methodology translates to other problems of explainable classification among a finite set of classes.

We present an iterative methodology using properties of the input data to the classification problem.  This supervised leaning methodology involves the discovery of domain-specific properties of the input data.  Transformation of the input data according to the properties.  Construction and training of distinct property-specific explainable Neural Networks (NNs).  Build an explainable confidence knowledgebase of results from training the property NNs.  The Explainability of eventual results comes from the property-specific NNs and the knowledgebase.  Classification can proceed according to the property NN output.  We pose a weighted voting mechanism based on the knowledgebase as well as an ML approach.   Properties are finally adjusted based on result scoring.

When applying this methodology to the classification problem in the MNIST handwritten digit database, there is a potential for acceptable classification while also providing a means of justification for classification decisions based on the explainable properties.

In addition to providing an explainable justification for classification decisions, multiple property-based neural networks in a wide arrangement may offer advantages in the simplicity (depth) of the NN architecture and gains in speed from the parallel operation of the distinct NNs.

\section{Related Work}

The ability to map the learning classifier or recognizer to human-based explainability is a challenging task for human understandability.  Currently, there are at least seventeen explainable techniques such as
decision tree-based, rule-based (i.e. knowledge-base), salience mapping,
sensitivity-based analysis, feature importance, fuzzy-based, neural-network, and generic-programming based.  These techniques use one of three basic evaluation approaches: application-grounded, human-grounded and functionally grounded. \cite{Hagras18} \cite{BlackBox18} \cite{Survey18} \cite{Fuzzy19} \cite{GP18}.

\section{Approach}

This methodology involves the following steps:
\begin{itemize}
\item Explainable property discovery.
\item Define data transformations according to properties.
\item Transform training data.
\item Produce trained explainable property-specific NNs.
\item Build a Confidence Knowledgebase across the explainable properties.
\item Classification using the Confidence Knowledgebase.
\item Using a test dataset to provide feedback.
\end{itemize}

These steps can be grouped into two phases.   The first phase involves the activities that result in producing the property-specific explainable trained NNs.  The second phase involves constructing the confidence knowledgebase and classification based on properties.  This sequence is depicted in Fig.~\ref{xaimeth} and described further in this section.

\begin{figure}[htbp]
\centerline{\includegraphics[width=50mm]{xai-approach0.png}}
\caption{Phase 1 of the XAI Methodology}
\label{xaimeth}
\end{figure}

The initial step in this phase is to discover explainable properties.  An explainable property is an attribute of a sample in the problem domain that may differentiate classes and provide a rationale for classification.  In the MNIST handwritten digit database, we pose explainable properties such as loops, lines, endpoints, and crossings.

Data transformation routines are next defined and implemented to represent a sample for classification according to the initial step's explainable properties.  A property may have one or more transforms.  Input to a transform is a sample to be classified, and the output of a transform routine will be the sample represented according to the property transform.  These transforms may be known algorithms of feature detection and extraction that may represent the properties or property characteristics.  In our example, we used digital image processing techniques related to the properties.  For example, a loop property may correspond to both closed or circular features. 

\begin{figure}[htbp]
\centerline{\includegraphics[width=100mm]{property-trans-training-set.png}}
\caption{Building Property Transform training sets}
\label{proptranstrain}
\end{figure}

We next generate a training dataset by submitting all elements from the original training set to the property transformations.  The output from each property transformation is stored separately for training each property transformation-specific NN.  The construction of the training sets for each property transformation for image $i$ is depicted in Fig. ~\ref{proptranstrain}, where the $i$th image is added to each collection of transformed training images.

The next step involves initializing unique NNs representing each property transformation and then using supervised ML techniques to train the NNs using only the output of that particular property transform and the original labels.  This results in a set of explainable property transforms and trained NNs that could be arranged in a pipeline as depicted in Fig.~\ref{widenn} that would allow an input image to be transformed and then classified according to each explainable property transform. 

\begin{figure}[htbp]
\centerline{\includegraphics[width=95mm]{property-nn.png}}
\caption{Property Transform trained NNs in a wide arrangement producing classification suggestions}
\label{widenn}
\end{figure}

After training, we utilize the pipeline depicted in Fig.~\ref{widenn} to process the training set and populate a confidence knowledgebase.  The knowledgebase stores the original training label, and each property transform classification from the property transform NNs.
We next devise a classification strategy using information from the knowledgebase.  We pose two in this work and suggest others for the future in TODO.
The first classification scheme is probabilistic.  In this scheme, we use the confidence knowledgebase to identify each NN's likelihood to classify the classes correctly.  Likelihoods are weighted and compared against the alternative classifications to provide confidence metrics for classifications.

After training, we utilize the pipeline depicted in Fig.~\ref{widenn} to process the training set and populate a confidence knowledgebase.  The knowledgebase stores the original training label, and each property transform classification from the property transform NNs.

\begin{figure}[htbp]
\centerline{\includegraphics[width=95mm]{property-nn-classification.png}}
\caption{The final classification pipeline with explainability}
\label{widennexp}
\end{figure}

We next devise a classification strategy using information from the knowledgebase.  The scheme will take as input the property transform classifications and output a final classification result as shown in Fig.~\ref{widennexp}.   The scheme must have the ability to tabulate votes from property transform NNs and intelligently select from potentially conflicting classifications.  We pose two schemes in this work and suggest others for the future in TODO.

The first classification scheme is probabilistic.  In this scheme, we use the confidence knowledgebase to identify each NN's likelihood of classifying the classes correctly.  Likelihoods are weighted, combined for like classifications, and then compared against the alternative classifications to provide confidence metrics for the potentially conflicting classifications.  This scheme may produce multiple classifications with confidence metrics if there is disagreement.

The second classification scheme utilizes the knowledgebase to train a NN to classify based on property transform results.  Using this scheme, we show very good results in our example.

Finally, when test data is presented to the pipeline, we evaluate the results and determine if more properties are needed.

Explanation for the final classification is provided in the property transform classification as depicted in Fig.~\ref{widennexp}.
 

\subsection{Maintaining the Integrity of the Specifications}

The IEEEtran class file is used to format your paper and style the text. All margins, 
column widths, line spaces, and text fonts are prescribed; please do not 
alter them. You may note peculiarities. For example, the head margin
measures proportionately more than is customary. This measurement 
and others are deliberate, using specifications that anticipate your paper 
as one part of the entire proceedings, and not as an independent document. 
Please do not revise any of the current designations.

\section{Prepare Your Paper Before Styling}
Before you begin to format your paper, first write and save the content as a 
separate text file. Complete all content and organizational editing before 
formatting. Please note sections \ref{AA}--\ref{SCM} below for more information on 
proofreading, spelling and grammar.

Keep your text and graphic files separate until after the text has been 
formatted and styled. Do not number text heads---{\LaTeX} will do that 
for you.

\subsection{Abbreviations and Acronyms}\label{AA}
Define abbreviations and acronyms the first time they are used in the text, 
even after they have been defined in the abstract. Abbreviations such as 
IEEE, SI, MKS, CGS, ac, dc, and rms do not have to be defined. Do not use 
abbreviations in the title or heads unless they are unavoidable.

\subsection{Units}
\begin{itemize}
\item Use either SI (MKS) or CGS as primary units. (SI units are encouraged.) English units may be used as secondary units (in parentheses). An exception would be the use of English units as identifiers in trade, such as ``3.5-inch disk drive''.
\item Avoid combining SI and CGS units, such as current in amperes and magnetic field in oersteds. This often leads to confusion because equations do not balance dimensionally. If you must use mixed units, clearly state the units for each quantity that you use in an equation.
\item Do not mix complete spellings and abbreviations of units: ``Wb/m\textsuperscript{2}'' or ``webers per square meter'', not ``webers/m\textsuperscript{2}''. Spell out units when they appear in text: ``. . . a few henries'', not ``. . . a few H''.
\item Use a zero before decimal points: ``0.25'', not ``.25''. Use ``cm\textsuperscript{3}'', not ``cc''.)
\end{itemize}

\subsection{Equations}
Number equations consecutively. To make your 
equations more compact, you may use the solidus (~/~), the exp function, or 
appropriate exponents. Italicize Roman symbols for quantities and variables, 
but not Greek symbols. Use a long dash rather than a hyphen for a minus 
sign. Punctuate equations with commas or periods when they are part of a 
sentence, as in:
\begin{equation}
a+b=\gamma\label{eq}
\end{equation}

Be sure that the 
symbols in your equation have been defined before or immediately following 
the equation. Use ``\eqref{eq}'', not ``Eq.~\eqref{eq}'' or ``equation \eqref{eq}'', except at 
the beginning of a sentence: ``Equation \eqref{eq} is . . .''

\subsection{\LaTeX-Specific Advice}

Please use ``soft'' (e.g., \verb|\eqref{Eq}|) cross references instead
of ``hard'' references (e.g., \verb|(1)|). That will make it possible
to combine sections, add equations, or change the order of figures or
citations without having to go through the file line by line.

Please don't use the \verb|{eqnarray}| equation environment. Use
\verb|{align}| or \verb|{IEEEeqnarray}| instead. The \verb|{eqnarray}|
environment leaves unsightly spaces around relation symbols.

Please note that the \verb|{subequations}| environment in {\LaTeX}
will increment the main equation counter even when there are no
equation numbers displayed. If you forget that, you might write an
article in which the equation numbers skip from (17) to (20), causing
the copy editors to wonder if you've discovered a new method of
counting.

{\BibTeX} does not work by magic. It doesn't get the bibliographic
data from thin air but from .bib files. If you use {\BibTeX} to produce a
bibliography you must send the .bib files. 

{\LaTeX} can't read your mind. If you assign the same label to a
subsubsection and a table, you might find that Table I has been cross
referenced as Table IV-B3. 

{\LaTeX} does not have precognitive abilities. If you put a
\verb|\label| command before the command that updates the counter it's
supposed to be using, the label will pick up the last counter to be
cross referenced instead. In particular, a \verb|\label| command
should not go before the caption of a figure or a table.

Do not use \verb|\nonumber| inside the \verb|{array}| environment. It
will not stop equation numbers inside \verb|{array}| (there won't be
any anyway) and it might stop a wanted equation number in the
surrounding equation.

\subsection{Some Common Mistakes}\label{SCM}
\begin{itemize}
\item The word ``data'' is plural, not singular.
\item The subscript for the permeability of vacuum $\mu_{0}$, and other common scientific constants, is zero with subscript formatting, not a lowercase letter ``o''.
\item In American English, commas, semicolons, periods, question and exclamation marks are located within quotation marks only when a complete thought or name is cited, such as a title or full quotation. When quotation marks are used, instead of a bold or italic typeface, to highlight a word or phrase, punctuation should appear outside of the quotation marks. A parenthetical phrase or statement at the end of a sentence is punctuated outside of the closing parenthesis (like this). (A parenthetical sentence is punctuated within the parentheses.)
\item A graph within a graph is an ``inset'', not an ``insert''. The word alternatively is preferred to the word ``alternately'' (unless you really mean something that alternates).
\item Do not use the word ``essentially'' to mean ``approximately'' or ``effectively''.
\item In your paper title, if the words ``that uses'' can accurately replace the word ``using'', capitalize the ``u''; if not, keep using lower-cased.
\item Be aware of the different meanings of the homophones ``affect'' and ``effect'', ``complement'' and ``compliment'', ``discreet'' and ``discrete'', ``principal'' and ``principle''.
\item Do not confuse ``imply'' and ``infer''.
\item The prefix ``non'' is not a word; it should be joined to the word it modifies, usually without a hyphen.
\item There is no period after the ``et'' in the Latin abbreviation ``et al.''.
\item The abbreviation ``i.e.'' means ``that is'', and the abbreviation ``e.g.'' means ``for example''.
\end{itemize}
An excellent style manual for science writers is \cite{b7}.

\subsection{Authors and Affiliations}
\textbf{The class file is designed for, but not limited to, six authors.} A 
minimum of one author is required for all conference articles. Author names 
should be listed starting from left to right and then moving down to the 
next line. This is the author sequence that will be used in future citations 
and by indexing services. Names should not be listed in columns nor group by 
affiliation. Please keep your affiliations as succinct as possible (for 
example, do not differentiate among departments of the same organization).

\subsection{Identify the Headings}
Headings, or heads, are organizational devices that guide the reader through 
your paper. There are two types: component heads and text heads.

Component heads identify the different components of your paper and are not 
topically subordinate to each other. Examples include Acknowledgments and 
References and, for these, the correct style to use is ``Heading 5''. Use 
``figure caption'' for your Figure captions, and ``table head'' for your 
table title. Run-in heads, such as ``Abstract'', will require you to apply a 
style (in this case, italic) in addition to the style provided by the drop 
down menu to differentiate the head from the text.

Text heads organize the topics on a relational, hierarchical basis. For 
example, the paper title is the primary text head because all subsequent 
material relates and elaborates on this one topic. If there are two or more 
sub-topics, the next level head (uppercase Roman numerals) should be used 
and, conversely, if there are not at least two sub-topics, then no subheads 
should be introduced.

\subsection{Figures and Tables}
\paragraph{Positioning Figures and Tables} Place figures and tables at the top and 
bottom of columns. Avoid placing them in the middle of columns. Large 
figures and tables may span across both columns. Figure captions should be 
below the figures; table heads should appear above the tables. Insert 
figures and tables after they are cited in the text. Use the abbreviation 
``Fig.~\ref{fig}'', even at the beginning of a sentence.

\begin{table}[htbp]
\caption{Table Type Styles}
\begin{center}
\begin{tabular}{|c|c|c|c|}
\hline
\textbf{Table}&\multicolumn{3}{|c|}{\textbf{Table Column Head}} \\
\cline{2-4} 
\textbf{Head} & \textbf{\textit{Table column subhead}}& \textbf{\textit{Subhead}}& \textbf{\textit{Subhead}} \\
\hline
copy& More table copy$^{\mathrm{a}}$& &  \\
\hline
\multicolumn{4}{l}{$^{\mathrm{a}}$Sample of a Table footnote.}
\end{tabular}
\label{tab1}
\end{center}
\end{table}

\begin{figure}[htbp]
\centerline{\includegraphics{fig1.png}}
\caption{Example of a figure caption.}
\label{fig}
\end{figure}

Figure Labels: Use 8 point Times New Roman for Figure labels. Use words 
rather than symbols or abbreviations when writing Figure axis labels to 
avoid confusing the reader. As an example, write the quantity 
``Magnetization'', or ``Magnetization, M'', not just ``M''. If including 
units in the label, present them within parentheses. Do not label axes only 
with units. In the example, write ``Magnetization (A/m)'' or ``Magnetization 
\{A[m(1)]\}'', not just ``A/m''. Do not label axes with a ratio of 
quantities and units. For example, write ``Temperature (K)'', not 
``Temperature/K''.

\section*{Acknowledgment}

The preferred spelling of the word ``acknowledgment'' in America is without 
an ``e'' after the ``g''. Avoid the stilted expression ``one of us (R. B. 
G.) thanks $\ldots$''. Instead, try ``R. B. G. thanks$\ldots$''. Put sponsor 
acknowledgments in the unnumbered footnote on the first page.

\section*{References}

Please number citations consecutively within brackets \cite{b1}. The 
sentence punctuation follows the bracket \cite{b2}. Refer simply to the reference 
number, as in \cite{b3}---do not use ``Ref. \cite{b3}'' or ``reference \cite{b3}'' except at 
the beginning of a sentence: ``Reference \cite{b3} was the first $\ldots$''

Number footnotes separately in superscripts. Place the actual footnote at 
the bottom of the column in which it was cited. Do not put footnotes in the 
abstract or reference list. Use letters for table footnotes.

Unless there are six authors or more give all authors' names; do not use 
``et al.''. Papers that have not been published, even if they have been 
submitted for publication, should be cited as ``unpublished'' \cite{b4}. Papers 
that have been accepted for publication should be cited as ``in press'' \cite{b5}. 
Capitalize only the first word in a paper title, except for proper nouns and 
element symbols.

For papers published in translation journals, please give the English 
citation first, followed by the original foreign-language citation \cite{b6}.

\begin{thebibliography}{00}
\bibitem{b1} G. Eason, B. Noble, and I. N. Sneddon, ``On certain integrals of Lipschitz-Hankel type involving products of Bessel functions,'' Phil. Trans. Roy. Soc. London, vol. A247, pp. 529--551, April 1955.
\bibitem{b2} J. Clerk Maxwell, A Treatise on Electricity and Magnetism, 3rd ed., vol. 2. Oxford: Clarendon, 1892, pp.68--73.
\bibitem{b3} I. S. Jacobs and C. P. Bean, ``Fine particles, thin films and exchange anisotropy,'' in Magnetism, vol. III, G. T. Rado and H. Suhl, Eds. New York: Academic, 1963, pp. 271--350.
\bibitem{b4} K. Elissa, ``Title of paper if known,'' unpublished.
\bibitem{b5} R. Nicole, ``Title of paper with only first word capitalized,'' J. Name Stand. Abbrev., in press.
\bibitem{b6} Y. Yorozu, M. Hirano, K. Oka, and Y. Tagawa, ``Electron spectroscopy studies on magneto-optical media and plastic substrate interface,'' IEEE Transl. J. Magn. Japan, vol. 2, pp. 740--741, August 1987 [Digests 9th Annual Conf. Magnetics Japan, p. 301, 1982].
\bibitem{b7} M. Young, The Technical Writer's Handbook. Mill Valley, CA: University Science, 1989.
\end{thebibliography}
\vspace{12pt}
\color{red}
IEEE conference templates contain guidance text for composing and formatting conference papers. Please ensure that all template text is removed from your conference paper prior to submission to the conference. Failure to remove the template text from your paper may result in your paper not being published.

\end{document}
